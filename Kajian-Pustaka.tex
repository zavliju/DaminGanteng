\chapter{Pembahasan}

\section{\emph{Preprocessing Data} dan Pendeteksian Anomali}

\subsection{\emph{Preprocessing Data}}
Contoh pustaka prosiding \cite{doyen2014explicit}, jurnal \cite{gunawan2015hydrostatic} dan buku \cite{toro2013riemann}. Atau dapat juga mengguanakan dua pustaka atau lebih dalam \cite{gunawan2015hydrostatic,toro2013riemann}.

\subsection{Pendeteksian Anomali}

\section{Penggunaan \emph{Tools} dan Analisis}

\subsection{Penggunaan \emph{Tools}}

\subsection{Analisis}
